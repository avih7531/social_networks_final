\documentclass[aspectratio=169, 11pt]{beamer}

% ============================================================
% THEME AND STYLING
% ============================================================
\usetheme{metropolis}
\usepackage{appendixnumberbeamer}

% Colors - Dark theme with accent colors
\definecolor{darkbg}{HTML}{1a1a2e}
\definecolor{darkfg}{HTML}{16213e}
\definecolor{accent}{HTML}{e94560}
\definecolor{accentblue}{HTML}{00d9ff}
\definecolor{textlight}{HTML}{eaeaea}
\definecolor{textsub}{HTML}{a0a0a0}

\setbeamercolor{normal text}{fg=textlight, bg=darkbg}
\setbeamercolor{alerted text}{fg=accent}
\setbeamercolor{example text}{fg=accentblue}
\setbeamercolor{frametitle}{fg=textlight, bg=darkfg}
\setbeamercolor{title}{fg=textlight}
\setbeamercolor{subtitle}{fg=textsub}
\setbeamercolor{author}{fg=textsub}
\setbeamercolor{date}{fg=textsub}
\setbeamercolor{block title}{fg=textlight, bg=darkfg}
\setbeamercolor{block body}{fg=textlight, bg=darkfg!50!darkbg}
\setbeamercolor{item}{fg=accent}
\setbeamercolor{subitem}{fg=accentblue}

% Progress bar
\metroset{progressbar=frametitle}
\setbeamercolor{progress bar}{fg=accent, bg=darkfg}

% ============================================================
% PACKAGES
% ============================================================
\usepackage{amsmath, amssymb}
\usepackage{graphicx}
\usepackage{booktabs}
\usepackage{tikz}
\usepackage{xcolor}
% \usepackage{fontspec}  % Commented out for pdflatex compatibility

% Path to diagrams
\graphicspath{{../diagrams/}}

% ============================================================
% TITLE INFO
% ============================================================
\title{Predicting Movie Genre Using\\Actor Co-Appearance Networks}
\subtitle{Can network topology alone reveal genre?}
\author{Avi Herman}
\date{Social Networks Final Project --- Fall 2025}
\institute{Department of Computer Science}

% ============================================================
% DOCUMENT
% ============================================================
\begin{document}

% ----------------------------------------------------------
% TITLE SLIDE
% ----------------------------------------------------------
\begin{frame}[plain]
  \maketitle
\end{frame}

% ----------------------------------------------------------
% MOTIVATION
% ----------------------------------------------------------
\begin{frame}{The Question}
  \begin{center}
    \Large
    \textcolor{accent}{Can we predict a movie's genre}\\[0.5em]
    \textcolor{accentblue}{using only actor collaborations?}\\[1.5em]
    \normalsize
    \textcolor{textsub}{No plot. No script. No text. No semantic features.}\\[0.5em]
    \textcolor{textsub}{Just \textbf{who worked with whom}.}
  \end{center}
\end{frame}

% ----------------------------------------------------------
% HYPOTHESIS
% ----------------------------------------------------------
\begin{frame}{Hypothesis}
  \begin{block}{Core Claim}
    Actor collaboration graphs exhibit strong \textbf{community structure}, and these communities correspond to \textbf{film genres} at rates significantly above random.
  \end{block}
  
  \vspace{1em}
  
  \textbf{Why might this work?}
  \begin{itemize}
    \item Directors repeatedly cast the same actors
    \item Production companies specialize in genres
    \item Franchise ecosystems create dense clusters
    \item Horror circuits, rom-com ensembles, prestige drama pools
  \end{itemize}
  
  \vspace{1em}
  
  \textcolor{accent}{\textbf{Genre may be a network-emergent property.}}
\end{frame}

% ----------------------------------------------------------
% CONCEPTUAL FRAMEWORK
% ----------------------------------------------------------
\begin{frame}{Conceptual Framework}
  \textbf{The Logic:}
  
  \vspace{1em}
  
  Actors specializing in similar film ecosystems tend to co-appear repeatedly. These recurring interactions create dense subgraphs that should map onto genre clusters if genre is a \textbf{socially emergent structure} within the film industry.
  
  \vspace{1.5em}
  
  \begin{block}{}
    \centering
    If communities correspond to genres, then \textcolor{accent}{genre is recoverable from network structure alone}.
  \end{block}
\end{frame}

\begin{frame}{Conceptual Framework (continued)}
  \textbf{We test this by:}
  \begin{enumerate}
    \item Constructing weighted co-appearance networks
    \item Detecting communities via modularity optimization
    \item Evaluating whether detected communities align with known genres
  \end{enumerate}
\end{frame}

% ----------------------------------------------------------
% NETWORK CONSTRUCTION
% ----------------------------------------------------------
\begin{frame}{Building the Network}
  \textbf{Graph Structure:} $G = (V, E, w)$
  \begin{itemize}
    \item \textbf{Nodes} $V = \{a_1, a_2, \ldots, a_n\}$ = Actors
    \item \textbf{Edges} $E = \{(a_i, a_j) : \text{co-appear in film}\}$
    \item \textbf{Weights} = Genre-purity weighting
  \end{itemize}
  
  \vspace{1.5em}
  
  \textbf{Edge Weight Function:}
  \[
  w(f) = \begin{cases}
  1.00 & \text{single-genre film} \\
  0.25 & \text{dual-genre} \\
  0.05 & \text{triple-genre} \\
  0.01 & \text{4+ genres}
  \end{cases}
  \]
\end{frame}

\begin{frame}{Building the Network (continued)}
  \textbf{Final edge weight between actors $a_i$ and $a_j$:}
  \[
  W_{ij} = \sum_{f \in \mathcal{F}(a_i, a_j)} w(f)
  \]
  
  where $\mathcal{F}(a_i, a_j)$ is the set of films featuring both actors.
  
  \vspace{1.5em}
  
  \begin{block}{}
    \centering
    \textcolor{textsub}{Intuition: A pure horror film is stronger evidence than Horror/Comedy/Romance.}
  \end{block}
\end{frame}

% ----------------------------------------------------------
% EDGE WEIGHT RATIONALE
% ----------------------------------------------------------
\begin{frame}{Edge Weight Rationale}
  \textbf{Why penalize multi-genre films?}
  
  \vspace{1em}
  
  \begin{columns}
    \column{0.5\textwidth}
    \textbf{Step-function weighting:}
    \begin{itemize}
      \item 1 genre: 1.00 (strong indicator)
      \item 2 genres: 0.25 (split signal)
      \item 3 genres: 0.05 (minimal)
      \item 4+ genres: 0.01 (nearly ignored)
    \end{itemize}
    
    \column{0.5\textwidth}
    \textbf{Rationale:}
    \begin{itemize}
      \item Single-genre films strongly indicate genre affinity
      \item Multi-genre films dilute genre specificity
      \item We want to prioritize genre-pure collaborations
    \end{itemize}
  \end{columns}
  
  \vspace{1.5em}
  
  \begin{block}{}
    \centering
    A collaboration in a pure horror film is \textcolor{accent}{much stronger evidence} of ``horror actor'' than a collaboration in a film tagged Horror/Comedy/Romance/Drama.
  \end{block}
\end{frame}

% ----------------------------------------------------------
% WHY COMMUNITIES
% ----------------------------------------------------------
\begin{frame}{Why Community Detection?}
  \textbf{The central premise:} Genre is encoded in collaboration structure
  
  \vspace{1em}
  
  \begin{block}{Key Insight}
    If actors cluster by genre, then:
    \begin{itemize}
      \item Actors within the same community share films more often than expected
      \item Each community should exhibit a dominant genre
      \item Community membership can serve as a genre \textbf{prediction}
    \end{itemize}
  \end{block}
  
  \vspace{1em}
  
  \textcolor{accent}{\textbf{Without community detection, we have no way to partition the network and test whether partitions correspond to genres.}}
\end{frame}

% ----------------------------------------------------------
% COMMUNITY DETECTION
% ----------------------------------------------------------
\begin{frame}{Community Detection: Louvain Algorithm}
  \textbf{Goal:} Find densely connected subgroups (communities)
  
  \vspace{1em}
  
  \textbf{Modularity:}
  \[
  Q = \frac{1}{2m} \sum_{i,j} \left[ A_{ij} - \frac{k_i k_j}{2m} \right] \delta(c_i, c_j)
  \]
  
  \vspace{1em}
  
  \begin{columns}
    \column{0.5\textwidth}
    \textbf{Symbols:}
    \begin{itemize}
      \item $A_{ij}$ = edge weight
      \item $k_i$ = weighted degree
      \item $m$ = total edge weight
      \item $c_i$ = community of $i$
    \end{itemize}
    
    \column{0.5\textwidth}
    \textbf{Interpretation:}
    \begin{itemize}
      \item $Q = 0$: Random
      \item $Q \approx 0.3$--$0.7$: Significant
      \item $Q > 0.7$: Strong
    \end{itemize}
  \end{columns}
\end{frame}

\begin{frame}{Community Detection: Louvain Algorithm (continued)}
  \textbf{Louvain Process:}
  \begin{enumerate}
    \item Each node greedily joins community maximizing $\Delta Q$
    \item Communities become super-nodes
    \item Repeat until convergence
  \end{enumerate}
  
  \vspace{1.5em}
  
  \begin{block}{}
    \centering
    Two-phase algorithm: \textbf{Local optimization} then \textbf{aggregation}
  \end{block}
\end{frame}

% ----------------------------------------------------------
% WHY MANY COMMUNITIES
% ----------------------------------------------------------
\begin{frame}{Why More Communities Than Genres?}
  The algorithm typically produces \textbf{more communities than the 5 macro-genres}
  
  \vspace{1em}
  
  \begin{columns}
    \column{0.5\textwidth}
    \textbf{Why?}
    \begin{itemize}
      \item Sub-genre specialization\\
        (slasher vs. supernatural horror)
      \item Franchise isolation\\
        (MCU actors form own cluster)
      \item Niche production circuits\\
        (low-budget regional films)
    \end{itemize}
    
    \column{0.5\textwidth}
    \textbf{This is informative!}
    \begin{itemize}
      \item Algorithm discovers natural structure
      \item Multiple resolution levels
      \item Not preset—determined by modularity optimization
    \end{itemize}
  \end{columns}
  
  \vspace{1em}
  
  \textcolor{accent}{\textbf{The number of communities $|\mathcal{C}|$ is determined by the data, not preset.}}
\end{frame}

% ----------------------------------------------------------
% WHAT COMMUNITIES CONTAIN
% ----------------------------------------------------------
\begin{frame}{What Do Communities Contain?}
  Each community $c \in \mathcal{C}$ is a set of actors:
  \[
  c = \{a_{i_1}, a_{i_2}, \ldots, a_{i_k}\}
  \]
  
  \vspace{1em}
  
  \textbf{Genre Distribution:} Characterize each community by its genre mix
  \[
  P_c(g) = \frac{\sum_{a \in c} \mathbb{1}[g^*(a) = g]}{|c|}
  \]
  
  where $g^*(a)$ is actor $a$'s dominant macro-genre.
  
  \vspace{1em}
  
  \begin{block}{}
    This gives a probability distribution over genres for each community.\\
    \textcolor{accent}{Pure communities} have one dominant genre.\\
    \textcolor{textsub}{Mixed communities} are spread across genres.
  \end{block}
\end{frame}

% ----------------------------------------------------------
% PURITY AND PREDICTION
% ----------------------------------------------------------
\begin{frame}{Community Purity}
  \textbf{Single Community Purity:}
  \[
  \text{Purity}(c) = \max_{g} \frac{|\{a \in c : g^*(a) = g\}|}{|c|}
  \]
  
  \vspace{1em}
  
  \textcolor{textsub}{What fraction of a community shares the same dominant genre?}
  
  \vspace{1.5em}
  
  \textbf{Overall Purity (Weighted):}
  \[
  \text{Purity}_{\text{overall}} = \frac{1}{|V|} \sum_{c \in \mathcal{C}} |c| \cdot \text{Purity}(c)
  \]
  
  \vspace{0.5em}
  
  \textcolor{textsub}{Equivalent to unweighted accuracy when assigning each community its majority genre.}
\end{frame}

\begin{frame}{Community Purity (continued)}
  \begin{columns}
    \column{0.5\textwidth}
    \textbf{Interpretation:}
    \begin{itemize}
      \item 0.20 = Random (5 genres)
      \item 0.40--0.50 = Weak clustering
      \item 0.60--0.70 = Moderate separation
      \item 0.80+ = Strong communities
    \end{itemize}
    
    \column{0.5\textwidth}
    \textcolor{accent}{\textbf{High purity} indicates that Louvain communities align with genre boundaries—the network \textit{knows} about genre even though genre labels were never used.}
  \end{columns}
\end{frame}

% ----------------------------------------------------------
% PURITY VS ACCURACY
% ----------------------------------------------------------
\begin{frame}{Purity vs. Accuracy}
  \begin{columns}
    \column{0.5\textwidth}
    \textbf{Purity}
    \begin{itemize}
      \item How homogeneous each community is
      \item Measures community-level consistency
      \item Equivalent to unweighted accuracy
    \end{itemize}
    
    \column{0.5\textwidth}
    \textbf{Accuracy}
    \begin{itemize}
      \item How well community assignment predicts individual actors
      \item Can exceed purity with weighting
      \item PageRank-weighting amplifies signal
    \end{itemize}
  \end{columns}
  
  \vspace{1.5em}
  
  \begin{block}{Key Point}
    With PageRank-weighting, accuracy can exceed raw purity because \textbf{central actors} (who define the community) tend to have cleaner genre profiles than peripheral actors.
  \end{block}
\end{frame}

% ----------------------------------------------------------
% ACCURACY FORMULAS
% ----------------------------------------------------------
\begin{frame}{Prediction Accuracy Metrics}
  \textbf{Three Voting Schemes:}
  
  \vspace{1em}
  
  \textbf{1. Unweighted Accuracy:}
  \[
  \text{Acc}_{\text{unweighted}} = \frac{1}{|V|} \sum_{a \in V} \mathbb{1}\left[ \hat{g}(c_a) = g^*(a) \right]
  \]
  Each actor contributes equally
  
  \vspace{1em}
  
  \textbf{2. Degree-Weighted Accuracy:}
  \[
  \text{Acc}_{\text{degree}} = \frac{\sum_{a \in V} k_a \cdot \mathbb{1}\left[ \hat{g}(c_a) = g^*(a) \right]}{\sum_{a \in V} k_a}
  \]
  Weights by collaboration volume
\end{frame}

\begin{frame}{Prediction Accuracy Metrics (continued)}
  \textbf{3. PageRank-Weighted Accuracy:}
  \[
  \text{Acc}_{\text{PR}} = \frac{\sum_{a \in V} \text{PR}(a) \cdot \mathbb{1}\left[ \hat{g}(c_a) = g^*(a) \right]}{\sum_{a \in V} \text{PR}(a)}
  \]
  Weights by network influence
  
  \vspace{1.5em}
  
  \begin{block}{Rationale}
    \centering
    Core actors (high PageRank) more reliably represent their community's genre identity than peripheral actors.
  \end{block}
\end{frame}

% ----------------------------------------------------------
% PAGERANK
% ----------------------------------------------------------
\begin{frame}{PageRank: Finding Core Actors}
  \[
  \text{PR}(a_i) = \frac{1-d}{N} + d \sum_{a_j \in \mathcal{N}(a_i)} \frac{W_{ji}}{\sum_k W_{jk}} \cdot \text{PR}(a_j)
  \]
  
  \vspace{0.5em}
  
  \begin{columns}
    \column{0.5\textwidth}
    \textbf{Parameters:}
    \begin{itemize}
      \item $d = 0.85$ (damping factor)
      \item $N$ = total actors
      \item $\mathcal{N}(a_i)$ = neighbors
    \end{itemize}
    
    \column{0.5\textwidth}
    \textbf{Intuition:}
    \begin{itemize}
      \item Actor is important if they collaborate with important actors
      \item Weighted by strength of collaborations
    \end{itemize}
  \end{columns}
  
  \vspace{1em}
  
  \begin{block}{Key Insight}
    \textbf{Core actors} (high PageRank) define community identity.\\
    \textbf{Peripheral actors} often appear in single cross-genre films.
  \end{block}
\end{frame}

% ----------------------------------------------------------
% GINI COEFFICIENT
% ----------------------------------------------------------
\begin{frame}{Gini Coefficient: Measuring Inequality}
  \textbf{Formula:}
  \[
  G = \frac{2 \sum_{i=1}^{n} i \cdot x_{(i)}}{n \sum_{i=1}^{n} x_{(i)}} - \frac{n+1}{n}
  \]
  
  where $x_{(i)}$ is the $i$-th smallest PageRank value.
  
  \vspace{1em}
  
  \begin{columns}
    \column{0.5\textwidth}
    \textbf{Interpretation:}
    \begin{itemize}
      \item $G \approx 0$: Perfect equality
      \item $G \approx 0.2$--$0.3$: Weak hub structure
      \item $G \approx 0.5$--$0.7$: Power-law (realistic)
      \item $G \rightarrow 1$: Extreme inequality
    \end{itemize}
    
    \column{0.5\textwidth}
    \textcolor{accent}{\textbf{What it tells us:}}
    \begin{itemize}
      \item Whether we have strong hubs
      \item How unequal actor importance is
      \item If network has realistic power-law shape
      \item If dataset is large enough to reveal structure
    \end{itemize}
  \end{columns}
\end{frame}

% ----------------------------------------------------------
% FALSIFICATION CRITERIA
% ----------------------------------------------------------
\begin{frame}{Falsification Criteria}
  \textbf{The hypothesis is rejected if:}
  
  \vspace{1em}
  
  \begin{columns}
    \column{0.33\textwidth}
    \centering
    \textcolor{accent}{\textbf{Modularity}}\\
    $Q < 0.3$ across all configurations
    
    \column{0.33\textwidth}
    \centering
    \textcolor{accentblue}{\textbf{Community Purity}}\\
    No dominant genre per community
    
    \column{0.33\textwidth}
    \centering
    \textcolor{accent}{\textbf{Accuracy}}\\
    Converges to random baseline $\approx 0.20$
  \end{columns}
  
  \vspace{2em}
  
  \begin{block}{}
    \centering
    \textbf{None of these occurred.}\\
    The hypothesis is strongly supported.
  \end{block}
\end{frame}

% ----------------------------------------------------------
% DATA
% ----------------------------------------------------------
\begin{frame}{Dataset}
  \textbf{Source:} IMDb Non-Commercial Datasets
  
  \vspace{1em}
  
  \begin{columns}
    \column{0.5\textwidth}
    \textbf{Filters:}
    \begin{itemize}
      \item Feature films only
      \item Runtime $\geq$ 59 min
      \item Released 1960+
      \item Top-3 billed cast
      \item Actors with $\geq$ 3 credits
      \item 18 canonical genres
    \end{itemize}
    
    \column{0.5\textwidth}
    \textbf{5 Macro-Genres:}
    \begin{itemize}
      \item \textcolor{accent}{ACTION}: Action, Adventure, Thriller, Sci-Fi, War
      \item \textcolor{accentblue}{DRAMA}: Drama, Romance, Biography
      \item \textcolor{accent}{COMEDY}: Comedy, Music, Musical
      \item \textcolor{accentblue}{DARK}: Crime, Mystery, Horror
      \item \textcolor{accent}{FAMILY}: Family, Animation, Fantasy
    \end{itemize}
  \end{columns}
  
  \vspace{1em}
  
  \textbf{Sample Sizes:} 250, 1000, and 5000 films (top by popularity)
\end{frame}

% ----------------------------------------------------------
% METHODOLOGY
% ----------------------------------------------------------
\begin{frame}{Experimental Design}
  \textbf{For each sample size $N \in \{250, 1000, 5000\}$:}
  
  \vspace{1em}
  
  \begin{enumerate}
    \item Select top-$N$ films by popularity (vote count)
    \item Construct weighted actor network
    \item Vary genre count from 3 to 12
    \item For each configuration:
    \begin{itemize}
      \item Compute Louvain partition (weighted and unweighted)
      \item Compute PageRank and Gini coefficient
      \item Evaluate all three accuracy metrics
    \end{itemize}
    \item Generate diagnostic visualizations
  \end{enumerate}
  
  \vspace{1em}
  
  \textcolor{accent}{\textbf{Total:} 3 sample sizes $\times$ 10 genre counts = 30 configurations}
\end{frame}

% ----------------------------------------------------------
% RESULTS: PAGERANK DISTRIBUTION
% ----------------------------------------------------------
\begin{frame}{Result 1: PageRank Distribution Evolution}
  \begin{center}
    \includegraphics[height=0.7\textheight]{across_sample_sizes/pagerank_distribution_comparison.png}
  \end{center}
  
  \vspace{0.5em}
  
  \begin{columns}
    \column{0.33\textwidth}
    \centering
    \textbf{250 films}\\
    Thin distribution\\
    No heavy tail
    
    \column{0.33\textwidth}
    \centering
    \textbf{1,000 films}\\
    Heavy tail starts\\
    Power-law forming
    
    \column{0.33\textwidth}
    \centering
    \textbf{5,000 films}\\
    Clear power-law\\
    Strong hubs
  \end{columns}
\end{frame}

% ----------------------------------------------------------
% RESULTS: GINI
% ----------------------------------------------------------
\begin{frame}{Result 1 (cont.): Network Inequality Scales with Size}
  \begin{columns}
    \column{0.55\textwidth}
    \includegraphics[width=\textwidth]{across_sample_sizes/gini_across_samples.png}
    
    \column{0.45\textwidth}
    \begin{tabular}{ccc}
      \toprule
      \textbf{Films} & \textbf{Actors} & \textbf{Gini} \\
      \midrule
      250 & 415 & 0.282 \\
      1,000 & 1,174 & 0.413 \\
      5,000 & 4,124 & 0.440 \\
      \bottomrule
    \end{tabular}
    
    \vspace{1em}
    
    \textcolor{accent}{\textbf{Finding:}}\\
    Larger datasets $\rightarrow$\\
    more realistic hub structure\\
    \textcolor{textsub}{(rich-get-richer)}
  \end{columns}
\end{frame}

% ----------------------------------------------------------
% RESULTS: MODULARITY
% ----------------------------------------------------------
\begin{frame}{Result 2: Strong Community Structure}
  \begin{columns}
    \column{0.55\textwidth}
    \includegraphics[width=\textwidth]{5000/modularity_original_genres.png}
    
    \column{0.45\textwidth}
    \begin{tabular}{cc}
      \toprule
      \textbf{Films} & \textbf{Peak $Q$} \\
      \midrule
      250 & \textbf{0.919} \\
      1,000 & \textbf{0.845} \\
      5,000 & \textbf{0.755} \\
      \bottomrule
    \end{tabular}
    
    \vspace{1em}
    
    \textcolor{accent}{\textbf{Finding:}}\\
    $Q > 0.7$ at all scales\\[0.5em]
    Actor networks are\\
    \textbf{highly modular}
  \end{columns}
\end{frame}

% ----------------------------------------------------------
% RESULTS: CONFUSION MATRIX
% ----------------------------------------------------------
\begin{frame}{Result 3: Genre Separation}
  \begin{columns}
    \column{0.5\textwidth}
    \includegraphics[width=\textwidth, height=0.65\textheight, keepaspectratio]{250/confusion_matrix_macro.png}
    
    \column{0.5\textwidth}
    \vspace{-0.5em}
    \textbf{Observations:}
    \begin{itemize}
      \item \textcolor{accent}{ACTION}: Sharply separated
      \item \textcolor{accentblue}{COMEDY/FAMILY}: Clean clusters
      \item \textcolor{textsub}{DRAMA}: Diffuse (meta-genre)
      \item DARK $\leftrightarrow$ ACTION overlap
    \end{itemize}
    
    \vspace{0.5em}
    
    \textcolor{accent}{\textbf{Communities align with genres!}}
  \end{columns}
\end{frame}

% ----------------------------------------------------------
% RESULTS: CO-OCCURRENCE
% ----------------------------------------------------------
\begin{frame}{Result 4: Genre Co-Occurrence Explains Overlaps}
  \begin{columns}
    \column{0.55\textwidth}
    \includegraphics[width=\textwidth]{5000/genre_cooccurrence_matrix.png}
    
    \column{0.45\textwidth}
    \textbf{Key Structures:}
    \begin{itemize}
      \item Action--Adventure--Thriller triad
      \item Family--Animation cluster
      \item Comedy avoids Dark
      \item Drama connects everything
    \end{itemize}
    
    \vspace{1em}
    
    \textcolor{textsub}{Multi-genre films create cross-community bridges.}
  \end{columns}
\end{frame}

% ----------------------------------------------------------
% RESULTS: NETWORK VIZ
% ----------------------------------------------------------
\begin{frame}{Result 5: Network Visualization}
  \begin{center}
    \includegraphics[height=0.75\textheight]{5000/actor_network.png}
  \end{center}
  \vspace{-0.5em}
  \centering\textcolor{textsub}{5,000 films. Node color = community. Node size = degree.}
\end{frame}

% ----------------------------------------------------------
% RESULTS: CHORD DIAGRAM
% ----------------------------------------------------------
\begin{frame}{Result 6: Genre Co-Occurrence Chord Diagram}
  \begin{columns}
    \column{0.6\textwidth}
    \includegraphics[width=\textwidth, height=0.85\textheight, keepaspectratio]{5000/genre_cooccurrence_chord.png}
    
    \column{0.4\textwidth}
    \textbf{Visualization:}
    \begin{itemize}
      \item Genres around circle
      \item Arcs = co-occurrence
      \item Thickness = frequency
    \end{itemize}
    
    \vspace{1em}
    
    \textcolor{accent}{\textbf{Key insight:}}\\
    Thick arcs $\rightarrow$ frequent co-occurrence $\rightarrow$ natural genre clusters
    
    \vspace{0.5em}
    
    \textcolor{textsub}{\small Example: Action--Adventure--Thriller triad}
  \end{columns}
\end{frame}

% ----------------------------------------------------------
% RESULTS: ACCURACY (THE BIG ONE)
% ----------------------------------------------------------
\begin{frame}{The Central Result: Prediction Accuracy}
  \begin{columns}
    \column{0.5\textwidth}
    \includegraphics[width=\textwidth]{across_sample_sizes/accuracy_across_samples.png}
    
    \column{0.5\textwidth}
    \begin{tabular}{ccc}
      \toprule
      \textbf{Films} & \textbf{Accuracy} & \textbf{Lift} \\
      \midrule
      250 & \textbf{76.0\%} & 3.8$\times$ \\
      1,000 & \textbf{63.9\%} & 3.2$\times$ \\
      5,000 & \textbf{61.0\%} & 3.1$\times$ \\
      \bottomrule
    \end{tabular}
    
    \vspace{0.5em}
    \textcolor{textsub}{Random baseline: 20\%}
    
    \vspace{1em}
    
    \textcolor{accent}{\textbf{3$\times$ better than random}}\\
    \textcolor{accent}{\textbf{using only topology!}}
  \end{columns}
\end{frame}

% ----------------------------------------------------------
% ACCURACY COMPARISON TABLE
% ----------------------------------------------------------
\begin{frame}{Accuracy Comparison: All Methods}
  \begin{center}
    \begin{tabular}{ccccc}
      \toprule
      \textbf{Films} & \textbf{Unweighted} & \textbf{Degree} & \textbf{PageRank} & \textbf{Random} \\
      \midrule
      250 & 0.720 & 0.696 & \textcolor{accent}{\textbf{0.760}} & 0.200 \\
      1,000 & 0.608 & 0.502 & \textcolor{accent}{\textbf{0.639}} & 0.200 \\
      5,000 & 0.524 & 0.553 & \textcolor{accent}{\textbf{0.610}} & 0.200 \\
      \bottomrule
    \end{tabular}
  \end{center}
\end{frame}

\begin{frame}{Accuracy Comparison: All Methods (continued)}
  \textbf{Key Findings:}
  \begin{enumerate}
    \item \textcolor{accent}{PageRank-weighted accuracy is consistently highest} across all sample sizes
    \item \textcolor{accentblue}{Accuracy exceeds 3$\times$ the random baseline} even at largest scale
    \item \textcolor{accent}{Central actors encode genre identity} more reliably than peripheral actors
    \item \textcolor{accentblue}{The network knows about genre} despite never being given genre labels
  \end{enumerate}
  
  \vspace{1em}
  
  \begin{block}{}
    \centering
    \textbf{Bottom line:} Using only network topology, we achieve \textcolor{accent}{\textbf{61.0\% accuracy}} on a 5-class problem where random guessing yields 20\%.
  \end{block}
\end{frame}

% ----------------------------------------------------------
% WHY IT WORKS
% ----------------------------------------------------------
\begin{frame}{Why Does Genre Emerge from Network Structure?}
  \textbf{The film industry operates through collaboration microcultures:}
  
  \vspace{1em}
  
  \begin{columns}
    \column{0.5\textwidth}
    \begin{itemize}
      \item Recurring casts (e.g., Christopher Nolan's ensemble)
      \item Director--actor partnerships
      \item Franchise ecosystems (MCU, horror sequels)
      \item Genre-bound production companies
    \end{itemize}
    
    \column{0.5\textwidth}
    \begin{itemize}
      \item Low-budget horror circuits
      \item Rom-com ensembles
      \item Prestige drama clusters
      \item Animation voice actor pools
    \end{itemize}
  \end{columns}
  
  \vspace{1.5em}
  
  \begin{block}{}
    \centering
    These \textbf{institutional structures} manifest as\\
    high-modularity subgraphs aligned with recognizable genres.
  \end{block}
\end{frame}

% ----------------------------------------------------------
% WHY ACCURACY DECLINES
% ----------------------------------------------------------
\begin{frame}{Why Does Accuracy Decline with Scale?}
  \textbf{Larger datasets introduce noise:}
  
  \vspace{1em}
  
  \begin{columns}
    \column{0.5\textwidth}
    \begin{itemize}
      \item \textcolor{accent}{Multi-genre actors}\\
        Blur community boundaries
      \item \textcolor{accentblue}{Cross-genre franchises}\\
        Create inter-community bridges
      \item \textcolor{accent}{High-degree bridge actors}\\
        Reduce modularity
    \end{itemize}
    
    \column{0.5\textwidth}
    \begin{itemize}
      \item \textcolor{accentblue}{DRAMA as umbrella genre}\\
        Connects disparate clusters
      \item \textcolor{textsub}{More genre mixing}\\
        Dilutes cluster purity
    \end{itemize}
  \end{columns}
  
  \vspace{1.5em}
  
  \begin{block}{}
    \centering
    \textbf{However, accuracy never approaches random}—\\
    genre signal persists at all scales.
  \end{block}
\end{frame}

% ----------------------------------------------------------
% WHY PAGERANK
% ----------------------------------------------------------
\begin{frame}{Why PageRank-Weighting Wins}
  \begin{block}{The Problem}
    Peripheral actors appear in single cross-genre films.\\
    They add \textbf{noise} to community genre assignment.
  \end{block}
  
  \vspace{1em}
  
  \begin{block}{The Solution}
    PageRank identifies \textbf{core actors} who define the community.\\
    These actors have consistent genre profiles.
  \end{block}
  
  \vspace{1em}
  
  \begin{center}
    \large
    \textcolor{accent}{PageRank weighting = amplify signal, suppress noise}
  \end{center}
\end{frame}

% ----------------------------------------------------------
% SUMMARY OF FINDINGS
% ----------------------------------------------------------
\begin{frame}{Summary of Findings}
  \vspace{0.5em}
  \begin{columns}
    \column{0.5\textwidth}
    \textbf{Network Properties:}
    \begin{itemize}
      \item \textcolor{accent}{Modularity:} $Q > 0.73$ at all scales
      \item \textcolor{accentblue}{Gini coefficient:} Increases with scale
      \item \textcolor{accent}{PageRank accuracy:} 3$\times$ random baseline
      \item \textcolor{accentblue}{Community purity:} Majority-genre assignment succeeds
    \end{itemize}
    \vspace{2.5em}
    \column{0.5\textwidth}
    \textbf{Key Results:}
    \begin{itemize}
      \item Strong community structure at every scale
      \item Realistic hub formation with larger datasets
      \item Clear genre clusters in network visualization
      \item PageRank-weighting consistently best method
    \end{itemize}
  \end{columns}
  
  \vspace{1.5em}
  
  \begin{block}{}
    \centering
    Actor co-appearance networks contain \textbf{substantial intrinsic genre information},\\
    recoverable without textual, plot, or semantic features.
  \end{block}
\end{frame}

% ----------------------------------------------------------
% LIMITATIONS
% ----------------------------------------------------------
\begin{frame}{Limitations}
  \begin{columns}
    \column{0.5\textwidth}
    \textbf{Methodological:}
    \begin{itemize}
      \item \textcolor{accent}{Genre aggregation} into 5 macro-genres may lose nuance
      \item \textcolor{accentblue}{Louvain algorithm} is greedy and may miss optimal partitions
      \item \textcolor{accent}{Edge weighting scheme} is heuristic-based
    \end{itemize}
    
    \column{0.5\textwidth}
    \textbf{Dataset:}
    \begin{itemize}
      \item \textcolor{accent}{IMDb bias} toward English-language, popular films
      \item \textcolor{accent}{Sample size limits} (max 5,000 films)
      \item \textcolor{accentblue}{Genre labels} may be inconsistent or subjective
    \end{itemize}
  \end{columns}
  
  \vspace{1.5em}
  
  \begin{block}{}
    \centering
    Despite limitations, results show \textcolor{accent}{\textbf{strong signal}} even with these constraints.
  \end{block}
\end{frame}

% ----------------------------------------------------------
% FUTURE WORK
% ----------------------------------------------------------
\begin{frame}{Future Work}
  \begin{columns}
    \column{0.5\textwidth}
    \textbf{Methodological Extensions:}
    \begin{itemize}
      \item \textcolor{accent}{Alternative algorithms}: Leiden, Infomap, spectral methods
      \item \textcolor{accentblue}{Temporal networks}: Track genre evolution over decades
      \item \textcolor{accent}{Multi-layer networks}: Directors, producers, studios
    \end{itemize}
    
    \column{0.5\textwidth}
    \textbf{Expanded Analysis:}
    \begin{itemize}
      \item \textcolor{accent}{International cinema}: Non-English language films
      \item \textcolor{accentblue}{Sub-genre detection}: Finer-grained classification
      \item \textcolor{accentblue}{Causal inference}: Does collaboration cause genre clustering?
    \end{itemize}
  \end{columns}
  
  \vspace{1.5em}
  
  \begin{block}{}
    \centering
    The network structure approach opens many directions for \textcolor{accent}{\textbf{further investigation}}.
  \end{block}
\end{frame}

% ----------------------------------------------------------
% CONCLUSION
% ----------------------------------------------------------
\begin{frame}{Conclusion}
  \begin{center}
    \Large
    \textcolor{accent}{\textbf{61\% accuracy}}\\[0.3em]
    \large on a 5-class problem\\[0.3em]
    \textcolor{textsub}{(random = 20\%)}\\[1.5em]
    
    \Large
    \textcolor{accentblue}{\textbf{Using only network structure}}\\[0.3em]
    \large No text. No plot. No semantics.
  \end{center}
  
  \vspace{1.5em}
  
  \begin{block}{}
    \centering
    Even at the largest and noisiest scale (5,000 films), the model achieves \textcolor{accent}{\textbf{61.0\% accuracy}} using only network structure—over three times the random baseline of 20\%.
  \end{block}
\end{frame}

\begin{frame}{Conclusion (continued)}
  \begin{block}{Theoretical Implications}
    Genre is not merely a content-based label. It is a \textcolor{accent}{network-emergent pattern} shaped by:
    \begin{itemize}
      \item Collaboration structure
      \item Institutional clustering
      \item Recurring partnerships
      \item Production company specialization
    \end{itemize}
  \end{block}
  
  \vspace{1.5em}
  
  \begin{center}
    \Large
    \textcolor{accentblue}{\textbf{Genre, fundamentally, is a pattern of relationships.}}
  \end{center}
\end{frame}

% ----------------------------------------------------------
% FINAL QUOTE
% ----------------------------------------------------------
\begin{frame}[plain]
  \begin{center}
    \vspace{3em}
    \Huge
    \textit{``Genre, fundamentally,}\\[0.3em]
    \textit{is a pattern of relationships.''}
    
    \vspace{3em}
    
    \normalsize
    \textcolor{textsub}{github.com/avih7531/social\_networks\_final}
  \end{center}
\end{frame}

\end{document}

